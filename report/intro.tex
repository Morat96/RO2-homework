\chapter{Introduction}

\section{Computational complexity theory}
First of all we want to start this report with a brief introduction to the computational complexity theory and in particular to the \textit{P} and \textit{NP}
classes. 
The computational complexity theory aim to establish how difficult is a certain problem. In order to do this, it is required a system to describe the complexity of the problem that has general validity, so that is independent from the computer that we are using. For this reason an algorithm is studied with respect to the number of elementary operations that are required to complete it if it's executed by a Turing machine. So for example if we say that an algorithm is $O(n)$ where n is the size of the instance, it means that it requires in the worst case a number of operation to finish that it's linear with respect to the problem size itself. \\
We need to spend a few words for the Turing machine. For simplicity we can consider a deterministic Turing machine (DTM) the equivalent of a deterministic computer and a nondeterministic Turing machine (NTM) the equivalent of an ideal nondeterministic computer that is able to exploit multiple action at the same time. Please notice that this is a great simplification, an in-depth discussion about Turing machine and the computational complexity theory exceeds the purpose of this introduction, that is only to give a general idea.
Now, in an informal way, we can say that a problem belong the the \textit{P} class if exist an algorithm that resolve it in a polynomial time on a deterministic Turing machine. Moreover we say that a problem belong to the \textit{NP} class (nondeterministic polynomial) if exist an algorithm able to resolve the problem in polynomial time on a nondeterministic Turing machine. \\
A DTM can simulate a NTM, but in the worst case this require an exponential number of operation (again there is a theorem that guarantee this, omitted on this report). This means that a problem that belong to the \textit{NP} class, can require an exponential number of operation to be resolved on a deterministic computer, making it infeasible if the instance is not small. \\
There are no theoretical proofs that $P \neq NP$ (This is one of the \textit{Millennium Prize Problems}) so this mean that NP problems may be exponential only because we don't know an efficient algorithm to resolve them, however nowadays this is considered unlikely and it's widely believed that some problems are intrinsically exponential to resolve.\\
In conclusion there are some problem that are considered even more difficult than the ones that belong to the \textit{NP} class. These are called \textit{NP-hard} and they are of particular interest because lots of problem that have practical application in reality belong to this typology. 

\section{Traveling salesman problem}
The traveling salesman problem 
